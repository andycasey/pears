% Copyright 2023 Andy Casey (Monash) and friends

\documentclass[modern]{aastex631}
\usepackage[utf8]{inputenc}
\usepackage{amsmath}

\renewcommand{\twocolumngrid}{}
\addtolength{\topmargin}{-0.35in}
\addtolength{\textheight}{0.6in}
\setlength{\parindent}{3.5ex}
\renewcommand{\paragraph}[1]{\medskip\par\noindent\textbf{#1}~---}

% figure setup
\usepackage{graphicx}
\usepackage{xcolor}
\usepackage[framemethod=tikz]{mdframed}
\usetikzlibrary{shadows}
\definecolor{captiongray}{HTML}{555555}
\mdfsetup{%
innertopmargin=2ex,
innerbottommargin=1.8ex,
linecolor=captiongray,
linewidth=0.5pt,
roundcorner=1pt,
shadow=false,
}
\newlength{\figurewidth}
\setlength{\figurewidth}{0.75\textwidth}

\newcommand{\norm}[1]{\left\lVert#1\right\rVert}

% Other possible titles
\newcommand{\chosentitle}{Pair-wise differential abundance measurements can yield extremely precise estimates of the chemical homogeneity of star clusters}

\shorttitle{Pair-wise differential abundance measurements}

\shortauthors{Casey}
\newcommand{\documentname}{\textsl{Article}}
\newcommand{\sectionname}{Section}

\newcommand{\project}[1]{\textit{#1}}
\renewcommand{\vec}[1]{\mathbf{#1}}
\newcommand{\vectheta}{\boldsymbol{\theta}}
\newcommand{\vecalpha}{\boldsymbol{\alpha}}
\newcommand{\vecbeta}{\boldsymbol{\beta}}
\newcommand{\vecgamma}{\boldsymbol{\gamma}}
\newcommand{\vecW}{\mathbf{W}} % stellar line absorption basis weights
\newcommand{\vecF}{\mathbf{F}} % stellar line absorption basis vectors
\newcommand{\vecG}{\mathbf{G}} % telluric line absorption basis vectors
\newcommand{\vecH}{\mathbf{H}} % continuum basis vectors
\newcommand{\vecX}{\mathbf{X}}

\newcommand{\hadamard}{\odot}
\newcommand{\apogee}{\project{APOGEE}}
\newcommand{\boss}{\project{BOSS}}
\newcommand{\sdss}{\project{SDSS}}
\newcommand{\eso}{\project{ESO}}
\newcommand{\harps}{\project{HARPS}}

\newcommand{\unit}[1]{\mathrm{#1}}
\newcommand{\mps}{\unit{m\,s^{-1}}}
\newcommand{\kmps}{\unit{km\,s^{-1}}}
\newcommand*{\transpose}{^{\mkern-1.5mu\mathsf{T}}}


\definecolor{tab:blue}{HTML}{1170aa}
\definecolor{tab:red}{HTML}{d1615d}
\newcommand{\todo}[1]{\textcolor{tab:red}{#1}}

\sloppy\sloppypar\raggedbottom\frenchspacing
\begin{document}

\title{\chosentitle}

\author[0000-0003-0174-0564]{Andrew R. Casey}
\affiliation{School of Physics \& Astronomy, Monash University, Australia}
\affiliation{Centre of Excellence for Astrophysics in Three Dimensions (ASTRO-3D)}
\affiliation{Center for Computational Astrophysics, Flatiron Institute, a division of the Simons Foundation}

\author{friends}


\begin{abstract}\noindent
    Stars are formed in clusters. Among stars that formed in the same cluster, there is some finite spread of chemical abundances. Knowing how this intrinsic abundance dispersion varies for different star clusters is important for understanding star formation, and places strong limits on how well we can infer the chemodynamic evolution of the Milky Way. Here we use basic statistics to show that the chemical homogeneity of star clusters can be estimated far more precisely from existing data, without the need for any reference (or so called benchmark) stars.
\end{abstract}



\section{Introduction} \label{sec:introduction}
Stellar spectroscopy is plagued by systematic effects. When measuring stellar metallicity from spectra, most analyses incorrectly assume the conditions are in local theromdynamic equilibrium, that a stellar photosphere is fully parameterised by one dimension, and an list of electron transitions that often have poorly known properties. These effects conspire to produce both biased and noisy estimates of metallicity.\\

These effects can be mitigated by measuring the chemical abundance of one star \emph{relative} to another. For two stars of similar stellar parameters, the effects of non-local thermodynamic equilibrium are approximately the same. If a single transition has poorly known parameters that lead to a biased estimate of metallicity, that bias is approximately the same for two very similar stars. This has led to the line-by-line differential abundance analysis technique, where all measurements are made relative to a so called reference star: where the stellar parameters of the reference star are well-measured by non-spectroscopic methods (e.g., asteroseismology, interferometry).\\


\section{Method} \label{sec:method}
Here we show that the line-by-line differential abundance technique can be extended to estimate the chemical homogenity of star clusters without the need for reference stars. Let ${X}$ be a set of observations that are drawn from a normal distribution
\begin{eqnarray}
   X \sim \mathcal{N}(\mu_x, \sigma_x^2)
\end{eqnarray}
\noindent{}with mean $\mu_x$ and variance $\sigma_x^2$. Here, $X$ represents the metallicity [Fe/H] measured from a star in a cluster, and we are interested in estimating $\sigma_x$. Because each estimate of $X$ is biased by systematic effects, we will instead measure the pair-wise metallicity $Y$
\begin{eqnarray}
    Y_{ij} = X_i - X_j
\end{eqnarray}
\noindent{}in a line-by-line differential manner. We provide a step-by-step example in Section~\ref{sec:experiments}. The variance of $Y_{ij}$ is given by
\begin{equation}
    \mathrm{Var}(Y_{ij}) = \mathrm{Var}(X_i - X_j)
\end{equation}
\begin{equation}
    \mathrm{Var}(Y_{ij}) = \mathrm{Var}(X_i) + \mathrm{Var}(X_j) - 2\mathrm{Cov}(X_i,X_j)
\end{equation}
\noindent{}where $\mathrm{Cov}(X_i,X_j)$ is the covariance between metallicity measurements of two (randomly selected) stars in a cluster that have near identical stellar parameters. We assume that the covariance will tend towards zero as the difference in stellar parameters of the two stars approaches zero. We further assume that the $X$ and $Y$ values are each identitically distributed such that 
\begin{equation}
    \mathrm{Var}(Y) = 2\,\mathrm{Var}(X)
\end{equation}
By taking many pair-wise abundance differences $Y$ and computing their sample variance $\mathrm{Var}(Y)$ we can estimate the intrinsic chemical homogenity of a cluster, and that estimate is largely unaffected by systematics due to stellar parameters and incorrect line properties:
\begin{equation}
    \sigma_x = \frac{\sigma_y}{\sqrt{2}} \quad .
    \label{eq:sigma_x_to_sigma_y}
\end{equation}

%The literature on pair-wise differential abundance measurements has shown that we can expect a precision in $Y$ of order 0.005 or 0.010\,dex, even for stars that we do not know for sure were born in the same cluster (e.g., just Solar analogues). To first order this would translate to being able to measure cluster homogenity to a precision between $\sigma_x = 0.003$ to 0.007~dex, about two orders of magnitude more precisely than literature estimates of chemical homogenity that are based entirely from sample variance of (systematically limited) classical measurements $\mathrm{Var}(X)$.

\section{Experiments} \label{sec:experiments}

Here simulate a star cluster with some intrinsic chemical homogeneity that we seek to estimate, where individual absorption lines have some smoothly-varying systematic differences as a function of stellar parameters. We take a 1 Gyr MIST isochrone with [Fe/H] = 0.03 and randomly select 100 evolutionary points between the main-sequence and the core-helium burning phase. We set the cluster mean abundance to be [Fe/H] = 0.03 dex to match our chosen isochrone, and we assume the cluster has an intrinsic metallicity dispersion of 0.003\,dex.\\

We idealistically assume that we could measure the same 250 iron lines in every star. For each line we will assume that it has some systematic bias that varies with effective temperature, where the bias is simulated by random draws from a Gaussian process that has a mean centered on zero. In other words, on average the lines will have no bias, but individually a line can have systematic biases that vary smoothly with effective temperature. Specifically we use a squared exponential kernel with a length scale of 3000\,K and a per line standard deviation of 0.3\,dex. We we will assume that there is random scatter in a single line abundance of 0.001\,dex, and that the scatter in effective temperature and surface gravity is 100\,K and 0.08\,dex, respectively.\\

The simulated data set is shown in the left panel of Figure~\ref{fig:simulation}. Stars have slightly different measured abundances, which tend to center around the true mean cluster abundance of 0.03\,dex. The effects of the per-line systematics are visible. Although our Gaussian process prior has zero mean, the mean star abundance (averaged over all lines) can demonstrate metallicity trends as a function of stellar parameters, which mimics results that are routinely reported from spectroscopy. In the right-hand panel of Figure~\ref{fig:simulation} we seek to estimate the chemical homogenity in two ways. The first is where we take the standard deviation of metallicity meausrements of individual stars. This statistic is dominated by systematic errors and leads to a significant over-estimate of the cluster chemical homogeneity ($0.007 \pm 0.001$\,dex). In the pair-wise estimate we have paired stars based on their stellar parameters, and computed the variance of the pair-wise abundance differences. Specifically, for each pair of stars we compute the mean of the $\Delta$[Fe/H] for each line. Then we compute the standard deviation of those pair-wise mean abundances, and multiply it by $1/\sqrt{2}$ to estimate $\sigma_{\mathrm{[Fe/H]}}$. The true level of cluster homogeniety is within the standard error of the pair-wise estimate.\\


\begin{figure*}
    \includegraphics*[width=\textwidth]{simulation.pdf}
    \caption{A simulation.\label{fig:simulation}}
\end{figure*}


\section{Discussion}


Let's say you have some stars in a cluster. Either you have observed them already or you haven't.
\begin{enumerate}
    \item Pair stars up. Either by their proximity in color-luminosity space, or by ordering a distance matrix of pair-wise things.
    \item For each star, estimate the stellar parameters in the way you know how. This could be by a classical excitation/ionization balance, spectrum fitting, etc, or by fixing the temperature from photometry and estimating logg from astrometry or an isochrone.
    \item For one atomic transition, compute the abundance in star A and star B. Note the difference $A-B$ and repeat it for all transitions.
    \item Take the mean of the line-by-line abundance differences as $Y_{ij}$. Keep the standard deviation. We will do something with it.
    \item Repeat steps 2-4 for all paired stars, then compute the sample variance. Times by $1/\sqrt{2}$ and write the paper.
\end{enumerate}



\section{Conclusions} \label{sec:conclusions}
We have shown that freshman statistics is useful.

\paragraph{Software}
\texttt{numpy} \citep{numpy}; 
\texttt{matplotlib} \citep{matplotlib}; 


% include bibliography
\bibliographystyle{aasjournal}
%\bibliography{bibliography}

\end{document}
